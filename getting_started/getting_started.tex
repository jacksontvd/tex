\documentclass{amsart}
\usepackage{master}
\begin{document}
\title{Getting started with \LaTeX\ and Vim}
\author{Jackson Van Dyke}
\date{\today}
\maketitle
\tableofcontents

I wrote this document to archive the process of setting up my workflow for taking notes
and writing papers in case I needed to reference it for some reason in the future. 
I figured there's no reason not to share it, in case someone might find it useful.
This is by no means exhaustive, and is meant as a jumping-off-point. The actual
documentation of the various software/packages should be the reader's next stop.
The \texttt{.tex} file for this document is available at 
\href{https://github.com/jacksontvd/tex/tree/master/getting_started}{this repository}.

If you just want my style files, these can found at 
\href{https://github.com/jacksontvd/tex}{this repository}, and if you just want my vimrc,
this can be found at \href{https://github.com/jacksontvd/vim}{this repository}.

This guide is written primarily for a mac, though most of what is written here would work
fine for any Unix machine.

\section{Introduction}

\LaTeX\ is a very simple typesetting tool which creates, in my opinion, very visually
appealing documents. Much of the generic appeal of \LaTeX\ is that a \texttt{.tex} document
comes in the form of a plaintext document which is somehow modular.
One sense in which this is true is that it can be compiled using any `document class'.
This means that if you decide to write a paper hoping to submit it to one journal which
wants the paper to look one way, and later decide to submit the paper to another journal
which wants it to look another way, the structure of the actual pdf document can be
changed immediately. This is just one example of the many things which make \LaTeX\ so
great. For more information and a much better technical introduction to how this thing
actually works, I suggest the canonical reference, \textit{The Texbook}, available
\href{http://www.ctex.org/documents/shredder/src/texbook.pdf}{here}.

\section{Setting up \LaTeX}

\subsection{Installation}

Download from, and follow instructions from \href{http://www.tug.org/mactex/}{this
webpage}.

\subsection{Macros}

\LaTeX\ macros should be thought of as custom `settings'. A basic example of such a thing
is the following:
\begin{verbatim}
\newcommand{\RR}{\mathbb{R}}
\end{verbatim}
which says that whenever one types \textbackslash\texttt{RR}
it is interpreted as \textbackslash\texttt{mathbb}\{\texttt{R}\}. All such rules are collected in what
is called a style file (a file with ending \texttt{.sty}) which lives in the directory 
\begin{verbatim}
~/Library/texmf/tex/latex/
\end{verbatim}
or the directory of whatever file you are editing. This is also where class files live.
See \href{https://github.com/jacksontvd/tex/tree/master/tex/latex}{this repository} for
mine.

\section{Setting up Vim}

\subsection{Installation}

Vim should come with your machine, but to make sure you have the latest version, or to
install it for the first time, it is easiest to use homebrew. I.e. enter the following to
the command line:
\begin{verbatim}
brew install vim
\end{verbatim}

\subsection{Writing a \texttt{vimrc}}

The content of a \texttt{vimrc} file should be thought of as custom settings for Vim.
When starting out it is tempting to copy and past someone else's. If you would like to do
this, mine can be found at \href{https://github.com/jacksontvd/vim}{this repository}. I
would however suggest building up your own gradually, adding things as you need them.
Google is helpful for this. 
By default your vimrc just lives in \texttt{/User/} but I prefer to have it in 
$\sim$\texttt{/.vim} so it is easier to push to by backup repository.

\subsection{Packages for Vim}

There are countless packages available for Vim. I mention the main ones I use. Consult the
documentation directly for more information.
\begin{enumerate}
\item Pathogen: In short, this package creates a directory such that whenever a package is
placed in it, the package is loaded. There are many alternatives to this. 
\item Nerd commenter: This package lets one comment lines quickly. It detects the document
type so one doesn't have to keep track of which symbols comment lines in which document
type, and instead one can have a designated keystroke combination which comments lines in
any language. 
\item Spell: Offers spell checking, and spell correction.
\item Ulti-Snips: Snippets are far too complicated to get into in depth, but they
basically allow you to type an abbreviation for something, hit a special key (by default
\texttt{<tab>}) and this abbreviation will expand to the larger predesignated thing. As
one can imagine, this is extremely useful for typing things quickly. Especially in \LaTeX.
To write and customize snippets:
\begin{verbatim}
:UltiSnipsEdit texmath
\end{verbatim}
\item Vimtex: This is, in my opinion, the best package supporting writing \LaTeX with Vim.
The alternative, Vim-LaTeX, is much heavier and isn't as flexible. In combination with
Ulti-Snips, vimtex can do much more. 
\item Vim-tex-fold: Supports folding of a \LaTeX document in Vim. This means that, for
example, sections will be collapsed to a single line on ones display when they are not
being edited. 
\end{enumerate}

\section{Displaying the PDF}

When editing notes quickly, it is useful to let the compiler run every time the document
is saved. This is the default mode in vim-tex. This means the PDF viewer one is using must
change as the file changes. Most PDF viewers do not support this. 
In my opinion there are only two effective options for PDF viewers on a mac that support
live updating: Skim and Zathura.

\subsection{Using Skim}

To install Skim enter the following into the command line:
\begin{center}
\begin{verbatim}
$ brew cask install skim
\end{verbatim}
\end{center}

\subsection{Using Zathura}

To tap, install, and link Zathura/the required plugins 
enter the following into the command line:
\begin{center}
\begin{verbatim}
$ brew tap zegervdv/zathura
$ brew install zathura --with-synctex
$ brew install zathura-pdf-poppler
$ brew install xdotool
$ mkdir -p $(brew --prefix zathura)/lib/zathura
$ ln -s $(brew --prefix zathura-pdf-poppler)/libpdf-poppler.dylib 
        $(brew --prefix zathura)/lib/zathura/libpdf-poppler.dylib
\end{verbatim}
\end{center}

\subsection{Making a choice}

If using vim-tex, once you have determined which viewer you would like to use, add one of the following
lines to your \texttt{vimrc}:
\begin{verbatim}
let g:vimtex_view_method='zathura'
let g:vimtex_view_method='skim'
\end{verbatim}
and the pdf file will automatically open in your chosen viewer when it is compiled.
Use \textbackslash\texttt{ll} to compile a document in vim-tex.
See the documentation of vim-tex for more information on PDF viewer support and compiling
documents.

\section{Writing a document}

This section assumes my \texttt{vimrc} and \LaTeX macros are being used.
The preamble and document environment (specifically for taking notes) is triggered by
\begin{verbatim}
notes<tab>
\end{verbatim}
with Ulti-Snips placeholders (toggled through with \texttt{<c-j>}) 
at all of the places where things should be added. 

To suggest a completion of a word:
\begin{verbatim}
<ctrl>n
\end{verbatim}

To correct spelling:
\begin{verbatim}
<ctrl>p
\end{verbatim}

To add a word to the dictionary:
\begin{verbatim}
zg
\end{verbatim}

To remove a word from the dictionary:
\begin{verbatim}
zug
\end{verbatim}

To reset folds:
\begin{verbatim}
\zx 
\end{verbatim}

To fold current fold:
\begin{verbatim}
\za
\end{verbatim}

Toggle whether current line is commented:
\begin{verbatim}
\c_
\end{verbatim}

Build an environment out of whatever is in the current line:
\begin{verbatim}
<ctrl>b
\end{verbatim}

\end{document}
